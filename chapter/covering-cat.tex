\subsection{Coverings of groupoids}
Much of the theory of covering spaces can be recast conceptually in terms of
fundamental groupoids. This point of view separates the essentials of the topology from the formalities and gives a convenient language in which to describe the algebraic classification of coverings.
\begin{definition}
Let $\mathcal{C}$ be a category and $x$ be an object of $C$. The category
$x\backslash\mathcal{C}$ of objects under $x$ is defined as follows.
\begin{itemize}
\item Objects of $x\backslash\mathcal{C}$ are maps $f:x\to y$ in $\mathcal{C}$
\item A morphism $\gamma$ between $f:x\to y$ and $g:x\to z$ is a morphism $\gamma: z\to y$ such that the followsing diagram commutes
\[\begin{tikzcd}
&x\ar[ld,swap,"f"]\ar[rd,"g"]&\\
z\ar[rr,"\gamma"]&&y
\end{tikzcd}\]
\end{itemize}
When $\mathcal{C}$ is a groupoid, $\gamma=g\circ f^{-1}$, and the objects of $x\backslash\mathcal{C}$ therefore determine the category.
\end{definition}
\begin{definition}
Let $\mathcal{C}$ be a small groupoid. Define the \textbf{star} of $x$, denoted $St(x)$ or $St_{\mathcal{C}}(x)$, to be the set of objects of $x\backslash\mathcal{C}$, that is, the set of morphisms of $\mathcal{C}$ with source $x$. Write $\pi(\mathcal{C},x):=\mathcal{C}(x,x)$ for the group of automorphisms of the object $x$.
\end{definition}
\begin{definition}
Let $\mathcal{E}$ and $\mathcal{B}$ be small connected groupoids. A \textbf{covering} $p:\mathcal{E}\to\mathcal{B}$ is a functor that is surjective on objects and restricts to a bijection
\[p:St(e)\to St(p(e))\]
for each object $e$ of $\mathcal{E}$. For an object $b$ of $\mathcal{B}$, let $F_b$ denote the set of objects of $\mathcal{E}$ such that $p(e)=b$. Then $p^{-1}(St(b))$ is the disjoint union over $e\in F_b$ of $St(e)$.
\end{definition}
Parts $(\rmnum{1})$ and $(\rmnum{2})$ of the unique path lifting theorem can be restated as follows
\begin{proposition}
If $p:E\to B$ is a covering of spaces, then the induced functor \[\Pi(p):\Pi(E)\to\Pi(B)\]
is a covering of groupoids.
\end{proposition}
Parts $(\rmnum{3})$, $(\rmnum{4})$, and $(\rmnum{5})$ of the unique path lifting theorem are categorical consequences that apply to any covering of groupoids, where they read as follows.
\begin{proposition}
Let $p:\mathcal{E}\to\mathcal{B}$ be a covering of groupoids, let $b$ be an object of $\mathcal{B}$, and let $e$ and $e'$ be objects of $F_b$.
\begin{itemize}
\item[$(\rmnum{1})$] $p:\pi(\mathcal{E},e)\to\pi(\mathcal{B},b)$ is a monomorphism.
\item[$(\rmnum{2})$] $p(\pi(\mathcal{E},e'))$ is conjugate to $p(\pi(\mathcal{E},e))$.
\item[$(\rmnum{3})$] As $e'$ runs through $F_b$, the groups $p(\pi(\mathcal{E},e'))$ run through all conjugates of $p(\pi(\mathcal{E},e))$ in $\pi(\mathcal{B},b)$.
\end{itemize}
\end{proposition}
\begin{proof}
For $(\rmnum{1})$, if $g,g'\in\pi(\mathcal{E},e)$ and $p(g)=p(g')$, then $g=g'$ by the injectivity of $p$ on $St(e)$.\par 
For $(\rmnum{2})$, there is a map $g:e\to e'$ since $\mathcal{E}$ is connected. Conjugation by $g$ gives a homomorphism $\pi(\mathcal{E},e)\to\pi(\mathcal{E},e')$ that maps under $p$ to conjugation
of $\pi(\mathcal{B},b)$ by its element $p(g)$.\par
For $(\rmnum{3})$, the surjectivity of $p$ on $St(e)$ gives that any $f\in\pi(\mathcal{B},b)$ is of the form $p(g)$ for some $g\in St(e)$. If $e'$ is the target of $g$, then $p(\pi(\mathcal{E},e'))$ is the conjugate of $p(\pi(\mathcal{E},e))$ by $f$.
\end{proof}
The fibers $F_b$ of a covering of groupoids are related by translation functions.
\begin{definition}
Let $p:\mathcal{E}\to\mathcal{B}$ be a covering of groupoids. Define the fiber
\textbf{translation functor} $T=T(p):\mathcal{B}\to\mathsf{Set}$ as follows. For an object $b$ of $\mathcal{B}$, $T(b)=F_b$. For a morphism $f:b\to b'$ of $\mathcal{B}$, $T(f):F_b\to F_{b'}$ is specified by $T(f)(e)=e'$, where $e'$ is the target of the unique $g$ in $St(e)$ such that $p(g)=f$.
\end{definition}
For a covering space $p:E\to B$ and a path $f:b\to b'$, $T(f):F_b\to F_{b'}$ is given by $T(f)(e)=g(1)$ where $g$ is the path in $E$ that starts at $e$ and covers $f$.
\begin{proposition}
Any two fibers $F_b$ and $F_{b'}$ of a covering of groupoids have the same cardinality. Therefore any two fibers of a covering of spaces have the same
cardinality.
\end{proposition}
\begin{proof}
For $f:b\to b'$, $T(f):F_b\to F_{b'}$ is a bijection with inverse $T(f^{-1})$.
\end{proof}
\subsection{Group actions and orbit categories}
A (left) action of a group $G$ on a set $S$ is a function $G\times S\to S$ such that $es=s$ (where $e$ is the identity element) and $(g'g)s=g'(gs)$ for all $s\in S$. The \textbf{isotropy group} $G_s$ of a point $s$ is the subgroup $\{g\mid gs=s\}$ of $G$, that is, the stablizer of $s$. An action is free if $gs=s$ implies $g=e$, that is, if $G_s=e$ for every $s\in S$.\par
The \textbf{orbit} generated by a point $s$ is $\{gs\mid g\in G\}$. An action is \textbf{transitive} if for every pair $s$, $s'$ of elements of $S$, there is an element $g$ of $G$ such that $gs=s'$. Equivalently, $S$ consists of a single orbit. If $H$ is a subgroup of $G$, the set $G/H$ of cosets $gH$ is a transitive $G$-set.\par 
When $G$ acts transitively on a set $S$, we obtain an isomorphism of $G$-sets between $S$ and the $G$-set $G/G_s$ for any fixed $s\in S$ by sending $gs$ to the coset $gG_s$.\par
The following lemma describes the group of automorphisms of a transitive $G$-set $S$. For a subgroup $H$ of $G$, let $N(H)$ denote the normalizer of $H$ in $G$ and define $W(H)=N(H)/H$. Such quotient groups $W(H)$ are sometimes called \textbf{Weyl groups}.
\begin{lemma}
Let $G$ act transitively on a set $S$, choose $s\in S$, and let $H=G_s$. Then $W(H)$ is isomorphic to the group $\Aut_G(S)$ of automorphisms of the $G$-set $S$.
\end{lemma}
\begin{proof}
For $n\in N(H)$ with image $[n]\in W(H)$, define an automorphism $\phi([n])$ of $S$ by $\phi([n])(gs)=gns$. For an automorphism $\phi$ of $S$, we have $\phi(s)=ns$ for some $n\in G$ since $G$ acts transitively. For $h\in H$, \[hns=\phi(hs)=\phi(s)=ns,\]
hence $n^{-1}hn\in G_s=H$ and $n\in N(H)$. It is easy to check that this bijection between $W(H)$ and $\Aut_G(S)$ is an isomorphism of groups.
\end{proof}
We shall also need to consider $G$-maps between different $G$-sets $G/H$.
\begin{lemma}\label{Weyl group auto}
A $G$-map $\alpha:G/H\to G/K$ has the form $\alpha(gH)=g\gamma K$, where the
element $\gamma\in G$ satisfies $\gamma^{-1}h\gamma\in K$ for all $h\in H$.
\end{lemma}
\begin{proof}
If $\alpha(eH)=\gamma K$, then the relation
\[\gamma K=\alpha(eH)=\alpha(hH)=h\alpha(H)=h\alpha(eH)=h\gamma K\]
implies that $\gamma^{-1}h\gamma\in K$ for all $h\in H$.
\end{proof}
\begin{definition}
The category $\mathcal{O}(G)$ of canonical orbits has objects the $G$-sets
$G/H$ and morphisms the $G$-maps of $G$-sets.
\end{definition}
The previous lemmas give some feeling for the structure of $\mathcal{O}(G)$ and lead to the following alternative description.
\begin{lemma}
The category $\mathcal{O}(G)$ is isomorphic to the category $\mathcal{G}$ whose objects are the subgroups of $G$ and whose morphisms are the distinct subconjugacy relations $\gamma^{-1}H\gamma\sub K$ for $\gamma\in G$.
\end{lemma}
If we regard $G$ as a category with a single object, then a (left) action of $G$ on a set $S$ is the same thing as a covariant functor $G\to\mathsf{Set}$. (A right action is the same thing as a contravariant functor.) If $\mathcal{B}$ is a small groupoid, it is therefore natural to think of a covariant functor $T:\mathcal{B}\to\mathsf{Set}$ as a generalization of a group action.
\begin{definition}
An action of a small groupoid $\mathcal{B}$ is a covariant functor $T:\mathcal{B}\to\mathsf{Set}$.
\end{definition}
For each object $b$ of $\mathcal{B}$, $T$ restricts to an action of $\pi(\mathcal{B},b)$ on $T(b)$. We say that the functor $T$ is transitive if this group action is \textbf{transitive} for each object $b$. If $\mathcal{B}$ is connected, this holds for all objects $b$ if it holds for any one object $b$.\par
For example, for a covering of groupoids $p:\mathcal{E}\to\mathcal{B}$, the fiber translation functor $T$ restricts to give an action of $\pi(\mathcal{B},b)$ on the set $F_b$. For $e\in F_b$, the isotropy group of $e$ is precisely $p(\pi(\mathcal{E},e))$. That is, $T(f)(e)=e$ if and only if the lift of $f$ to an element of $St(e)$ is an automorphism of $e$. Moreover, the action is transitive since there is an isomorphism in $\mathcal{E}$ connecting any two points of $F_b$. Therefore, as a $\pi(\mathcal{B},b)$-set,
\[F_b\simeq\pi(\mathcal{B},b)/p(\pi(\mathcal{E},e))\]
\begin{definition}
A covering $p:\mathcal{E}\to\mathcal{B}$ of groupoids is \textbf{regular} if $p(\pi(\mathcal{E},e))$ is a normal subgroup of $\pi(\mathcal{B},b)$. It is \textbf{universal} if $p(\pi(\mathcal{E},e))=\{e\}$. Clearly a covering space is regular or universal if and only if its associated covering of fundamental groupoids is regular or universal.
\end{definition}
A covering of groupoids is universal if and only if $\pi(\mathcal{B},b)$ acts freely on $F_b$, and then $F_b$ is isomorphic to $\pi(\mathcal{B},b)$ as a $\pi(\mathcal{B},b)$-set. Specializing to covering spaces, this sharpens our earlier claim that the elements of $F_b$ and $\pi_1(\mathcal{B},b)$ are in bijective correspondence.
\subsection{The classification of coverings of groupoids}
Fix a small connected groupoid $\mathcal{B}$ throughout this section and the next. We explain the classification of coverings of $\mathcal{B}$. This gives an algebraic prototype for the classification of coverings of spaces. We begin with a result that should be called the fundamental theorem of covering groupoid theory. We assume once and for all that all given groupoids are small and connected.
\begin{theorem}
Let $p:\mathcal{E}\to\mathcal{B}$ be a covering of groupoids, let $X$ be a groupoid, and let $f:\mathcal{X}\to\mathcal{B}$ be a functor. Choose a base object $x_0\in\mathcal{X}$, let $b_0=f(x_0)$, and choose $e_0\in F_{b_0}$. Then there exists a functor $g:\mathcal{X}\to\mathcal{E}$ such that $g(x_0)=e_0$
and $p\circ g=f$ if and only if
\[f(\pi(\mathcal{X},x_0)\sub p(\pi(\mathcal{E},e_0))\]
in $\pi(\mathcal{B},b_0)$. When this condition holds, there is a unique such functor $g$, and we have a commutative diagram
\[\begin{tikzcd}
\mathcal{E}\ar[rr,"p"]&&\mathcal{B}\\
&\mathcal{X}\ar[lu,dashed,"\exists!g"]\ar[ru,swap,"f"]&
\end{tikzcd}\]
\end{theorem}
\begin{proof}
If $g$ exists, its properties directly imply that $\im(f)\sub\im(p)$. For an
object $x$ of $\mathcal{X}$ and a map $\alpha:x_0\to x$ in $\mathcal{X}$, let $\widetilde{\alpha}$ be the unique element of $St(e_0)$ such that $p(\widetilde{\alpha})=f(\alpha)$. If $g$ exists, $g(\alpha)$ must be $\widetilde{\alpha}$ and therefore $g(x)$ must be the target $T(f(\alpha))(e_0)$ of $\widetilde{\alpha}$. The inclusion $f(\pi(\mathcal{X},x_0))\sub p(\pi(\mathcal{E},e_0))$ ensures that $T(f(\alpha))(e_0)$ is independent of the choice of $\alpha$, so that $g$ so specified is a well defined functor. In fact, given another map $\alpha':x_0\to x$, $\alpha^{-1}\alpha'$ is an element of $\pi(\mathcal{X},x_0)$. Therefore
\[f(\alpha)^{-1}\circ f(\alpha')=f(\alpha^{-1}\circ\alpha')=p(\beta)\]
for some $\beta\in\pi(\mathcal{E},e_0)$. Thus
\[p(\widetilde{\alpha}\circ\beta)=f(\alpha)\circ p(\beta)=f(\alpha)\circ f(\alpha)^{-1}\circ f(\alpha')\]
This means that $\widetilde{\alpha}\circ\beta$ is the unique element $\widetilde{\alpha}'$ of $St(e_0)$ such that $p(\widetilde{\alpha}')=f(\alpha')$, and its target is the target of $\widetilde{\alpha}$, as required.
\end{proof}
\begin{definition}
A map $g:\mathcal{E}\to\mathcal{E}'$ of coverings of $\mathcal{B}$ is a functor $g$ such that the following diagram of functors is commutative:
\[\begin{tikzcd}
\mathcal{E}\ar[rd,"p"]\ar[rr,"g"]&&\mathcal{E}'\ar[ld,swap,"p'"]\\
&\mathcal{B}&
\end{tikzcd}\]
Let $\mathrm{Cov}(\mathcal{B})$ denote the category of coverings of $\mathcal{B}$; when $\mathcal{B}$ is understood, we write $\mathrm{Cov}(\mathcal{E},\mathcal{E}')$ for the set of maps $\mathcal{E}\to\mathcal{E}'$ of coverings of $\mathcal{B}$.
\end{definition}
\begin{lemma}
A map $g:\mathcal{E}\to\mathcal{E}'$ of coverings is itself a covering.
\end{lemma}
\begin{proof}
The functor $g$ is surjective on objects since, if $e'\in E'$ and we choose an object $e\in\mathcal{E}$ and a map $f:g(e)\to e'$ in $\mathcal{E}'$, and consider the map $p'(f):p'(g(e))\to p'(e')$ in $\mathcal{B}$. Since $p'(g(e))=p(e)$, we lift $p'(f)$ by $p$ to get a map $\widetilde{f}:e\to x$ in $\mathcal{E}$. Now $\widetilde{f}$ is maped to $g(\widetilde{f}):g(e)\to g(x)$ in $\mathcal{E}'$, and we have
\[p(x)=p'(e'),\quad p'(g(x))=p(x)=p'(e')\]
By the bijectivity of $p'$ on $St_{\mathcal{E}'}(g(e))$, we conclude $g(x)=e'$, so $e'=g(T_p(p'(f))(e))$ as needed. The map $g:St_{\mathcal{E}}(e)\to St_{\mathcal{E}'}(g(e))$ is a bijection since its composite with the bijection $p':St_{\mathcal{E}'}(g(e))\to St_{\mathcal{B}}(p'(g(e)))$ is the bijection $p:St_{\mathcal{E}}(e)\to St_{\mathcal{B}}(p(e))$.
\end{proof}
The fundamental theorem immediately determines all maps of coverings of $\mathcal{B}$ in terms of group level data.
\begin{theorem}\label{fundamental theo}
Let $p:\mathcal{E}\to\mathcal{B}$ and $p':\mathcal{E}'\to\mathcal{B}$ be coverings and choose base objects $b\in\mathcal{B}$, $e\in\mathcal{E}$, and $e'\in\mathcal{E}'$ such that $p(e)=b=p'(e')$. There exists a map $g:\mathcal{E}\to\mathcal{E}'$ of coverings with $g(e)=e'$ if and only if
\[p(\pi(\mathcal{E},e))\sub p'(\pi(\mathcal{E}',e'))\]
and there is then only one such $g$. In particular, two maps of covers $g$, $g':\mathcal{E}\to\mathcal{E}'$ coincide if $g(e)=g'(e)$ for any one object $e\in\mathcal{E}$. Moreover, $g$ is an isomorphism if and only if the displayed inclusion of subgroups of $\pi(\mathcal{B},b)$ is an equality. Therefore $\mathcal{E}$ and $\mathcal{E}'$ are isomorphic if and only if $p(\pi(\mathcal{E},e))$ and $p'(\pi(\mathcal{E}',e'))$ are conjugate whenever $p(e)=p'(e')$.
\end{theorem}
\begin{proof}
We only prove the last claim. Suppose $p(\pi(\mathcal{E},e))=\tau p'(\pi(\mathcal{E}',e'))\tau^{-1}$ for some element $\tau\in\pi(\mathcal{B},x)$ where $x=p(e)=p'(e')$. Let $\lambda:e'\to e''$ be a map in $\mathcal{E}'$ such that $p'(\lambda)=\tau$. Then we have 
\[\lambda\pi(\mathcal{E}',e')\lambda^{-1}=\pi(\mathcal{E}',e'').\]
Therefore, it follows that
\[p'(\pi(\mathcal{E}',e''))=p'(\lambda\pi(\mathcal{E}',e')\lambda^{-1})=\tau p'(\pi(\mathcal{E}',e'))\tau^{-1}=p(\pi(\mathcal{E},e))\]
so we get an isomorphism.
\end{proof}
\begin{corollary}
If it exists, the universal cover of $\mathcal{B}$ is unique up to isomorphism
and covers any other cover.
\end{corollary}
\begin{theorem}\label{rest to fiber}
Let $p:\mathcal{E}\to\mathcal{B}$ and $p':\mathcal{E}'\to\mathcal{B}$ be coverings, choose a base object $b\in\mathcal{B}$, and let $G=\pi(\mathcal{B},b)$. If $g:\mathcal{E}\to\mathcal{E}'$ is a map of coverings, then $g$ restricts to a map $F_b\to F'_b$ of $G$-sets, and restriction to fibers specifies a bijection between $\mathrm{Cov}(\mathcal{E},\mathcal{E}')$ and the set of $G$-maps $F_b\to F'_b$.
\end{theorem}
\begin{proof}
Let $e\in F_b$ and $f\in\pi(\mathcal{B},b)$. By definition, $f(e)$ is the target of the map $\widetilde{f}\in St_{\mathcal{E}}(e)$ such that $p(\widetilde{f})=f$. Clearly $g(f(e))$ is the target of $g(\widetilde{f})\in St_{\mathcal{E}'}(g(e))$ and $p'(g(\widetilde{f}))=p(\widetilde{f})=f$. Again by definition, this gives $g(f(e))=f(g(e))$. It is clear that restriction to fibers is an injection on $\mathrm{Cov}(\mathcal{E},\mathcal{E}')$. To show surjectivity, let $\alpha:F_b\to F'_b$ be a $G$-map. Choose $e\in F_b$ and let $e'=\alpha(e)$. Since $\alpha$ is a $G$-map, the isotropy group $p(\pi(\mathcal{E},e))$ of $e$ is contained in the isotropy group $p'(\pi(\mathcal{E}',e'))$ of $e'$. Therefore Theorem~\ref{fundamental theo} ensures the existence of a covering map $g$ that restricts to $\alpha$ on fibers.
\end{proof}
\begin{definition}
Let $\Aut(\mathcal{E})\sub\mathrm{Cov}(\mathcal{E},\mathcal{E})$ denote the group of automorphisms of a cover $\mathcal{E}$. Note that, since it is possible to have conjugate subgroups $H$ and $H'$ of a group $G$ such that $H$ is a proper subgroup of $H'$, it is possible to have a map of covers $g:\mathcal{E}\to\mathcal{E}$ such that $g$ is not an isomorphism. $($Review the proof in Theorem~\ref{fundamental theo}$)$
\end{definition}
\begin{corollary}
Let $p:\mathcal{E}\to\mathcal{B}$ be a covering and choose objects $b\in\mathcal{B}$ and $e\in F_b$. Write $G=\pi(\mathcal{B},b)$ and $H=p(\pi(\mathcal{E},e))$. Then $\Aut(\mathcal{E})$ is isomorphic to the group of automorphisms of the $G$-set $F_b$ and therefore to the group $W(H)$. If $p$
is regular, then $\Aut(\mathcal{E})\simeq G/H$. If $p$ is universal, then $\Aut(\mathcal{E})\simeq G$.
\end{corollary}
\subsection{The construction of coverings of groupoids}
\begin{theorem}
Choose a base object $b$ of $\mathcal{B}$ and let $G=\pi(\mathcal{B},b)$. There is a functor
\[\mathscr{E}:\mathcal{O}(G)\to\mathrm{Cov}(\mathcal{B})\]
that is an equivalence of categories. For each subgroup $H$ of $G$, the covering $p:\mathscr{E}(G/H)\to\mathcal{B}$ has a canonical base object $e$ in its fiber over $b$ such that
\[p(\pi(\mathscr{E}(G/H),e))=H.\]
Moreover, $F_b=G/H$ as a $G$-set and, for a $G$-map $\alpha:G/H\to G/K$ in $\mathcal{O}(G)$, the restriction of $\mathscr{E}(\alpha):\mathscr{E}(G/H)\to \mathscr{E}(G/K)$ to fibers over $b$ coincides with $\alpha$.
\end{theorem}
\begin{proof}
The idea is that, up to bijection, $St_{\mathcal{E}(G/H)}(e)$ must be the same set for each $H$, but the nature of its points can differ with $H$, and this is realized by identifying morphisms in $\mathcal{B}$ at different level depending on $H$.\par
For a subgroup $H$, we define a covering of $\mathcal{B}$ as follows
\begin{itemize}
\item The groupoid $\mathscr{E}(G/H)$ is defined to be $St_{\mathcal{B}}(b)/H$ where the quotient is taken with respcet to the composition in $\mathcal{B}$ since $\mathcal{B}$ is a groupoid. That is, for two morphisms $f$ and $f'$,
\[f\sim f'\iff f^{-1}\circ f'\in H\sub\pi(\mathcal{B},b).\]
And let $e$ be the coset of the identity morphism $id_b$, that is, clearly $e:=H$.
\item We define the morphism sets in $\mathscr{E}(G/H)$ to be
\[\Hom_{\mathscr{E}(G/H)}(fH,f'H)=\{f'\circ h\circ f^{-1}\mid h\in H\}.\]
Composition and identities are inherited from those of $\mathcal{B}$. For $f_1H=f_2H$, from the definition we have
\begin{align*}
\Hom_{\mathscr{E}(G/H)}(f_1H,f'H)&=f'\circ H\circ f_1^{-1}=f'\circ H\circ f_1^{-1}\circ f_2\circ f_2^{-1}\\
&=f'\circ H\circ f^{-1}_2=\Hom_{\mathscr{E}(G/H)}(f_2H,f'H)
\end{align*}
since $f_1^{-1}\circ f_2\in H$. So the Hom sets are well defined.
\item Define a functor $p:\mathscr{E}(G/H)\to\mathcal{B}$ by assigning each object $fH$ in $St_{\mathcal{B}}(b)/H$ the target of $f$ and each morphisms the inclusion from
\[\Hom_{\mathscr{E}(G/H)}(fH,f'H)\sub\Hom_{\mathcal{B}}(p(fH),p(f'H)).\]
This makes no ambiguity since equivalent morphisms $f$, $f'$ should have the same target so that we can compose $f^{-1}$ and $f'$.
\end{itemize}
The surjectivity of $p$ comes from the connectivity of $\mathcal{B}$, and the covering property comes from the observation that every morphism $\alpha:p(fH)\to y$ in $St_{\mathcal{B}}(p(fH))$ gives a morphism
\[(\alpha\circ f)\circ id_b\circ f^{-1}\in\Hom_{\mathscr{E}(G/H)}(fH,(\alpha\circ f)H)\]
and conversely every morphism in $St_{\mathscr{E}(G/H)}(fH)$ gives a morphism in $St_{\mathcal{B}}(p(fH))$ from the inclusion. Since this gives a bijection, the functor $p$ is a covering. And it is clear that $p(\pi(\mathscr{E}(G/H),e))=H$.\par
This defines the object function of the functor $\mathscr{E}:\mathcal{O}(G)\to\mathrm{Cov}(\mathcal{B})$. To define $\mathcal{E}$ on morphisms, consider $\alpha:G/H\to G/K$. If $\alpha(eH)=gK$, then $g^{-1}Hg\sub K$ and $\alpha(fH)=fgK$ (Lemma~\ref{Weyl group auto}). The functor $\mathscr{E}(\alpha):\mathscr{E}(G/H)\to\mathscr{E}(G/K)$ sends the object
$fH$ to the object $\alpha(fH)=fgK$ and sends the morphism $f'\circ h\circ f^{-1}$ to the same morphism of $\mathcal{B}$ regarded as 
\[f'g\circ g^{-1}hg\circ g^{-1}f^{-1}.\]
It is easily checked that each $\mathscr{E}(\alpha)$ is a well defined functor, and that $\mathscr{E}$ is functorial in $\alpha$.\par
To show that the functor $\mathscr{E}$ is an equivalence of categories, it suffices to show that it maps the morphism set $\Hom_{\mathcal{O}(G)}(G/H,G/K)$ bijectively onto the morphism set $\mathrm{Cov}(\mathscr{E}(G/H),\mathscr{E}(G/K))$ and that every covering of $\mathcal{B}$ is isomorphic to one of the coverings $\mathscr{E}(G/H)$. These statements are immediate from the results of Theorem~\ref{fundamental theo} and ~\ref{rest to fiber}.
\end{proof}
The following remarks place the orbit category $\mathcal{O}(\pi(\mathcal{B},b))$ in perspective by relating it to several other equivalent categories.
\begin{remark}
Consider the category $\mathsf{Set}^{\mathcal{B}}$ of functors $T:\mathclap{B}\to\mathsf{Set}$ and natural transformations. Let $G=\pi(\mathcal{B},b)$. Regarding $G$ as a category with one object $b$, it is a skeleton of $\mathcal{B}$, hence the inclusion $G\sub\mathcal{B}$ is an equivalence of categories.\par
Therefore, restriction of functors $T$ to $G$-sets $T(b)$ gives an equivalence of categories from $\mathsf{Set}^{\mathcal{B}}$ to the category of $G$-sets. This restricts to an equivalence between the respective subcategories of transitive objects. We have chosen to focus on transitive objects since we prefer to insist that coverings be connected. The inclusion of the orbit category $\mathcal{O}(G)$ in the category of transitive $G$-sets is an equivalence of categories because $\mathcal{O}(G)$ is a full subcategory that contains a skeleton. We could shrink $\mathcal{O}(G)$ to a skeleton by choosing one $H$ in each conjugacy class of subgroups of $G$, but the resulting equivalent subcategory is a less natural mathematical object.
\end{remark}
\subsection{The classification of coverings of spaces}
We begin with the following result, which deserves to be called the fundamental theorem of covering space theory and has many other applications. It asserts that the fundamental group gives the only obstruction to solving a certain lifting problem. Recall our standing assumption that all given spaces are connected and locally path connected.
\begin{theorem}
Let $p:E\to B$ be a covering and let $f:X\to B$ be a continuous map. Choose $x\in X$, let $b=f(x)$, and choose $e\in F_b$. There exists a map $g:X\to E$ such that $g(x)=e$ and $p\circ g=f$ if and only if
\[f_*(\pi_1(X,x))\sub p_*(\pi_1(E,e))\]
in $\pi_1(B,b)$. When this condition holds, there is a unique such map $g$.
\end{theorem}
\begin{proof}
If $g$ exists, its properties directly imply that $\im(f_*)\sub\im(p_*)$. Thus
assume that $\im(f_*)\sub\im(p_*)$. Applied to the covering $\Pi(p):\Pi(E)\to\Pi(B)$, the analogue for groupoids gives a functor $\Pi(X)\to\Pi(E)$ that restricts on objects to the unique map $g:X\to E$ of sets such that $g(x)=e$ and $p\circ g=f$.\par
We need only check that $g$ is continuous, and this holds because $p$ is a local homeomorphism. In detail, if $y\in X$ and $g(y)\in U$, where $U$ is an open subset of $E$, then there is a smaller open neighborhood $U'$ of $g(y)$ that $p$ maps homeomorphically onto an open subset $V$ of $B$. If $W$ is any path connected neighborhood of $y$ such that $f(W)\sub V$, then $g(W)\sub U'$ by inspection of the definition of $g$.
\end{proof}
\begin{definition}
A map $g:E\to E'$ of coverings over $B$ is a map $g$ such that the following diagram is commutative:
\[\begin{tikzcd}
E\ar[rd,"p"]\ar[rr,"g"]&&E'\ar[ld,swap,"p'"]\\
&B&
\end{tikzcd}\]
Let $\mathrm{Cov}(B)$ denote the category of coverings of the space $B$; when $B$ is understood, we write $Cov(E,E')$ for the set of maps $E\to E'$ of coverings of $B$.
\end{definition}
\begin{lemma}
A map $g:E\to E'$ of coverings is itself a covering.
\end{lemma}
\begin{proof}
The map $g$ is surjective by the algebraic analogue. The fundamental neighborhoods for $g$ are the components of the inverse images in $E'$ of the neighborhoods of $B$ which are fundamental for both $p$ and $p'$.
\end{proof}
\begin{theorem}
Let $p:E\to B$ and $p':E'\to B$ be coverings and choose $b\in B$, $e\in E$, and $e'\in E'$ such that $p(e)=b=p'(e')$. There exists a map $g:E\to E'$ of
coverings with $g(e)=e'$ if and only if
\[p_*(\pi_1(E,e))\sub p'_*(\pi_1(E',e'))\]
and there is then only one such $g$. In particular, two maps of covers $g, g':E\to E'$ coincide if $g(e)=g'(e)$ for any one $e\in E$. Moreover, $g$ is a homeomorphism if and only if the displayed inclusion of subgroups of $\pi_1(B,b)$ is an equality. Therefore $E$ and $E'$ are homeomorphic if and only if $_*(\pi_1(E,e))$ and $p'_*(\pi_1(E',e'))$ are conjugate whenever $p(e)=p'(e')$.
\end{theorem}
\begin{corollary}
If it exists, the universal cover of $B$ is unique up to isomorphism and covers any other cover.
\end{corollary}
\begin{corollary}
The fundamental groupoid functor induces a bijection
\[\mathrm{Cov}(E,E')\to\mathrm{Cov}(\Pi(E),\Pi(E'))\]
\end{corollary}
Just as for groupoids, we can recast the theorem in terms of fibers. In fact,
via the corollary, the following result is immediate from its analogue for groupoids.
\begin{theorem}
Let $p:E\to B$ and $p':E'\to B$ be coverings, choose a basepoint $b\in B$, and let $G=\pi_1(B,b)$. If $g:E\to E'$ is a map of coverings, then $g$ restricts to a map $F_b\to F'_b$ of $G$-sets, and restriction to fibers specifies a bijection between $\mathrm{Cov}(E,E')$ and the set of $G$-maps $F_b\to F'_b$.
\end{theorem}
\begin{definition}
Let $\Aut(E)\sub\mathrm{Cov}(E,E)$ denote the group of automorphisms of a cover $E$. Again, just as for groupoids, it is possible to have a map of covers
$g:E\to E$ such that $g$ is not an isomorphism.
\end{definition}
\begin{corollary}
Let $p:E\to B$ be a covering and choose $b\in B$ and $e\in F_b$. Write $G=\pi_1(B,b)$ and $H=p_*(\pi_1(E,e))$. Then $\Aut(E)$ is isomorphic to the
group of automorphisms of the $G$-set $F_b$ and therefore to the group $W(H)$. If $p$ is regular, then $\Aut(E)\simeq G/H$. If $p$ is universal, then $\Aut(E)\simeq G$.
\end{corollary}
\subsection{The construction of coverings of spaces}
We have now given an algebraic classification of all possible covers of $B$: there is one isomorphism class of covers corresponding to each conjugacy class
of subgroups of $\pi_1(B,b)$. We show here that all of these possibilities are actually realized. We shall first construct universal covers and then show that the existence of universal covers implies the existence of all other possible covers. Again, while it suffices to think in terms of locally contractible spaces, appropriate generality demands a weaker hypothesis.
\begin{definition}
We say that a space $B$ is \textbf{semi-locally simply connected} if every point $b\in B$ has a neighborhood $U$ such that $\pi_1(U,b)\to\pi_1(B,b)$ is the trivial homomorphism.
\end{definition}
\begin{theorem}
If $B$ is connected, locally path connected, and semi-locally simply connected, then $B$ has a universal cover.
\end{theorem}
\begin{proof}
Fix a base point $b\in B$. We turn the properties of paths that must hold in a universal cover into a construction. Define $E$ to be the set of equivalence
classes of paths $f$ in $B$ that start at $b$ and define $p:E\to B$ by $p[f]=f(1)$.\par 
Of course, the equivalence relation is homotopy through paths from $b$ to a given endpoint, so that $p$ is well defined. Thus, as a set, $E$ is just $St_{\Pi(B)}(b)$, exactly as in the construction of the universal cover of $\Pi(B)$.\par 
The topology of $B$ has a basis $\mathcal{U}$ consisting of path connected open subsets $U$ such that $\pi_1(U,u)\to\pi_1(B,u)$ is trivial for all $u\in U$. Since every loop in $U$ is equivalent in $B$ to the trivial loop, any two paths $u\to u'$ in such a $U$ are equivalent in $B$. We shall topologize $E$ so that $p$ is a cover with these $U$ as fundamental neighborhoods. For a path $f$ in $B$ that starts at $b$ and ends in $U$, define a subset $U_{[f]}$ of $E$ by
\[U_{[f]}=\{[g]\mid [g]=[f\ast c]\text{ for some }c:I\to U\}\]
A simple observation is that 
\[f'\in U_{[f]}\Rightarrow U_{[f]}=U_{[f']}.\]
The set of all such $U_{[f]}$ is a basis for a topology on $E$ since if $U_{[f]}$ and $U'_{[f']}$ are two such sets and $[g]$ is in their intersection, then
\[W_{[g]}\in U_{[f]}\cap U'_{[f']}\]
for any open set $W$ of $B$ such that $p([g])\in W\sub U\cap U'$.\par
For $u\in U$, there is a unique $[g]$ in each $U_{[f]}$ such that $p([g])=u$ since $\pi_1(U,u)\to\pi_1(B,u)$ is trivial so every two paths having same end points are homotopic. Further, every $u$ has a preimage since $U$ is path connected. Thus $p$ is a bijection. The restriction $p|_U:U_{[f]}\to U$ is a homeomorphism since it gives a bijection between the subsets $V_{[f']}\sub U_{[f]}$ and the sets $V\in\mathcal{U}$ contained in $U$. Namely, in one direction we have $p(V_{[f']})=V$ and in the other direction we have $p^{-1}(V)\cap U_{[f]}=V_{[f']}$ for any $[f']\in U_{[f]}$ with endpoint in $V$, since $V_{[f']}\sub U_{[f']}=U_{[f]}$ and $V_{[f']}$ maps onto $V$ by the bijection $p$. This also shows $p:E\to B$ is comtinuous. We can also deduce
that this is a covering space since for fixed $U\in\mathcal{U}$, the sets $U_{[f]}$ for varying $[f]$ partition $p^{-1}(U)$ because if $[f'']\in U_{[f]}\cap U_{[f']}$ then $U_{[f]}=U_{[f'']}=U_{[f']}$.\par 
It only remains to show that $E$ is simply connected. For a point $[f]\in E$, let $f_t$ be the path in $X$ that equals $f$ on $[0,t]$ and is stationary at $f(t)$ on $[t,1]$. Then the function $t\mapsto[f_t]$ is a path in $E$ lifting $f$ that starts at $[c_b]$, the homotopy class of the constant path at $b$, and ends at $[f]$. Since $[f]$ was an arbitrary point in $E$, this shows that $E$ is path-connected. To show that $\pi_1(E,b)=0$ it suffices to show that the image of this group under $p_*$ is trivial since $p_*$ is injective. Elements in the image of $p∗_*$ are represented by loops $f$ at $b$ that lift to loops in $E$ at $[c_b]$. We have observed that the path $t\mapsto[f_t]$ lifts $f$ starting at $[c_b]$, and for this lifted path to be a loop means that $[f_1]=[c_b]$. Since $f_1=f$, this says that $[f]=[c_b]$, so $f$ is nullhomotopic and the image of $p_*$ is trivial.
\end{proof}
We shall construct general covers by passage to orbit spaces from the universal
cover, and we need some preliminaries.
\begin{definition}
A $G$-space $X$ is a space $X$ that is a $G$-set with continuous action map $G\times X\to X$. Define the orbit space $X\backslash G$ to be the set of orbits $\{Gx\mid x\in X\}$ with its topology as a quotient space of $X$.
\end{definition}
The definition makes sense for general topological groups $G$. However, our
interest here is in discrete groups $G$, for which the continuity condition just means that action by each element of $G$ is a homeomorphism. The functoriality on $\mathcal{O}(G)$ of our construction of general covers will be immediate from the following observation.
\begin{lemma}
Let $X$ be a $G$-space. Then passage to orbit spaces defines a functor $X\backslash(-):\mathcal{O}(G)\to\mathsf{Top}$.
\end{lemma}
\begin{proof}
The functor sends $G/H$ to $X/H$ and sends a map $\alpha:G/H\to G/K$ to the map $X/H\to X/K$ that sends the coset $Hx$ to the coset $K\gamma^{-1}x$, where $\alpha$ is given by the subconjugacy relation $\gamma^{-1}H\gamma\sub K$.
\end{proof}
The starting point of the construction of general covers is the following description of regular covers and in particular of the universal cover.
\begin{proposition}
Let $p:E\to B$ be a cover such that $\Aut(E)$ acts transitively on $F_b$. Then the cover $p$ is regular and $E/\Aut(E)$ is homeomorphic to $B$.
\end{proposition}
\begin{proof}
For any points $e,e'\in F_b$, there exists $g\in\Aut(E)$ such that $g(e)=e'$ and thus $p_*(\pi_1(E,e))=p_*(\pi_1(E,e′))$. Therefore all conjugates of $p_*(\pi_1(E,e))$ are equal to $p_*(\pi_1(E, e))$ and $p_*(\pi_1(E,e))$ is a normal subgroup of $\pi_1(B,b)$. The homeomorphism is clear since, locally, both $p$ and passage to orbits identify the different components of the inverse images of fundamental neighborhoods.
\end{proof}
\begin{theorem}
Choose a basepoint $b\in B$ and let $G=\pi_1(B,b)$. There is a functor
\[E:\mathcal{O}(G)\to\mathrm{Cov}(B)\]
that is an equivalence of categories. For each subgroup $H$ of $G$, the covering $p:E(G/H)\to B$ has a canonical basepoint $e$ in its fiber over $b$ such that
\[p_*(\pi_1(E(G/H),e))=H\]
Moreover, $F_b\simeq G/H$ as a $G$-set and, for a $G$-map $\alpha:G/H\to G/K$ in $\mathcal{O}(G)$, the restriction of $E(\alpha):E(G/H)\to E(G/K)$ to fibers over $b$ coincides with $\alpha$.
\end{theorem}
\begin{proof}
Let $p:E\to B$ be the universal cover of $B$ and fix $e\in E$ such that $p(e)=b$. We have the isomorphism $\Aut(E)\simeq\pi_1(B,b)$ given by mapping
$g:E\to E$ to the path class $[f]\in G$ such that $g(e)=T(f)(e)$, where $T(f)(e)$ is the endpoint of the path $\widetilde{f}$ that starts at $e$ and lifts $f$. We identify subgroups of $G$ with subgroups of $\Aut(E)$ via this isomorphism. We define $E(G/H)$ to be the orbit space $E/H$ and we let $q:E\to E/H$ be the quotient map. We may identify $B$ with $E/\Aut(E)$, and inclusion of orbits specifies a map $p':E/H\to B$ such that $p'\circ q=p:E\to B$. If $U\sub B$ is a fundamental neighborhood for $p$ and $V$ is a component of $p^{-1}(U)\sub E$, then
\[p^{-1}(U)=\coprod_{g\in\Aut(E)}gV.\]
Passage to orbits over $H$ simply identifies some of these components, and we see immediately that both $p'$ and $q$ are covers. If $e'=q(e)$, then $p'_*$ maps $\pi_1(E/H,e')$ isomorphically onto $H$ since, by construction, the isotropy group of $e'$ under the action of $\pi_1(B,b)$ is precisely $H$. Rewriting $p'=p$ and $e'=e$ generically, this gives the stated properties of the coverings $E(G/H)$. The functoriality on $\mathcal{O}(G)$ follows directly from the previous lemma.\par
The functor $E(-)$ is an equivalence of categories since the results that it maps the morphism set $\mathcal{O}(G)(G/H,G/K)$ bijectively onto the morphism set $\mathrm{Cov}(E(G/H),E(G/K))$ and that every covering of $B$ is isomorphic to one of the coverings $E(G/H)$.
\end{proof}
The classification theorems for coverings of spaces and coverings of groupoids
are nicely related. In fact, the following diagram of functors commutes up to natural isomorphism:
\[\begin{tikzcd}
&\mathcal{O}(\pi_1(B,b))\ar[ldd,swap,"E"]\ar[rdd,"\mathscr{E}"]\\
&&\\
\mathrm{Cov}(B)\ar[rr,"\Pi"]&&\mathrm{Cov}(\Pi(B))
\end{tikzcd}\]
\begin{corollary}
$\Pi:\mathrm{Cov}(B)\to\mathrm{Cov}(\Pi(B))$ is an equivalence of categories.
\end{corollary}
\subsection{Exercise}
\begin{exercise}
In the following two problems, let $G$ be a connected and locally path connected topological group with identity element $e$, let $p:H\to G$ be a covering, and fix $f\in H$ such that $p(f)=e$. Prove the following.
\begin{itemize}
\item[$(1)$]
\begin{itemize}
\item[$(a)$] $H$ has a unique continuous product $H\times H\to H$ with identity
element $f$ such that $p$ is a homomorphism.
\item[$(b)$] $H$ is a topological group under this product, and $H$ is Abelian if $G$ is.
\end{itemize}
\item[$(2)$]
\begin{itemize}
\item[$(a)$] The kernel $K$ of $p$ is a discrete normal subgroup of $H$.
\item[$(b)$] In general, any discrete normal subgroup $K$ of a connected topological group $H$ is contained in the center of $H$.
\item[$(c)$] For $k\in K$, define $t(k):H\to H$ by $t(k)(h)=kh$. Then $k\to t(k)$ specifies an isomorphism between $K$ and the group $\Aut(H)$.
\end{itemize}
\end{itemize}
\end{exercise}
