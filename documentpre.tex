\usepackage[utf8]{inputenc}
\usepackage[colorlinks,linkcolor=blue,hyperfootnotes=false]{hyperref}
\usepackage{graphicx}
\usepackage{multicol}
\usepackage{amsfonts}
\usepackage{amsmath,amssymb,amsthm}
\usepackage{textcomp}
\usepackage{extpfeil}
\usepackage{geometry}
\usepackage{wasysym}
\usepackage{extarrows}
\usepackage[bottom]{footmisc}
\usepackage{enumitem}
\usepackage{textgreek,bm,upgreek,mathrsfs}
\usepackage{stmaryrd}
\usepackage{epigraph}
\usepackage[many]{tcolorbox}
\usepackage[english]{babel}
\usepackage{mathdesign}
\usepackage{setspace}
\usepackage{microtype} 
\usepackage[framemethod=tikz]{mdframed} 
\usepackage[tikz]{bclogo}
\usepackage{lipsum}
\usepackage{wrapfig}
\usepackage{color}
\usepackage{epigraph}
\usepackage[many]{tcolorbox}
\usepackage{tikz-cd}
\usepackage{tikz}
\usepackage{fancyhdr}
\usepackage[capitalize,nameinlink]{cleveref}%动态ref定理,推论,引理等

\usepackage[T1]{fontenc}
\PassOptionsToPackage{no-math}{fontspec}

\usepackage[noBBpl]{mathpazo}

%\usepackage{libertine}
%\usepackage{palatino}
%\usepackage{utopia}
%\usepackage{newcent}

\usepackage{bbold}%定义bbold符号
\usepackage{l3draw,xparse}
\usepackage{accents}%创建新的widetilde
\usepackage{scalerel} %放大数学字体
\usepackage{calligra,mathrsfs}%for sheaf Hom
\usepackage{mathtools}%def new widetilde
\usepackage{calc}%定义箭头不同样式symbol
\usepackage{ulem}%定义合适的下划线

%-----------------------------------------------------
%定义BOONDOX包的字体
\DeclareMathAlphabet{\mathbbb}{U}{bbold}{m}{n}
\DeclareMathAlphabet{\mathbb}{U}{msb}{m}{n}
%-----------------------------------------------------
\DeclareFontFamily{U}{BOONDOX-cal}{\skewchar\font=45}
\DeclareFontShape{U}{BOONDOX-cal}{m}{n}{
  <-> s*[1] BOONDOX-r-cal}{}
\DeclareMathAlphabet{\mathscr}{U}{BOONDOX-cal}{m}{n}
%-----------------------------------------------------
\DeclareFontFamily{U}{BOONDOX-frak}{\skewchar\font=45}
\DeclareFontShape{U}{BOONDOX-frak}{m}{n}{
  <-> s*[1]  BOONDOX-r-frak}{}
%\DeclareMathAlphabet{\mathfrak}{U}{BOONDOX-frak}{m}{n}

%----------------------------------------------------
%设置目录中number和文字的间距
\usepackage{tocloft}
\setlength{\cftsecnumwidth}{25pt} % resets space for section number (usually 2.3em)
\setlength{\cftsubsecnumwidth}{35pt} % resets space for subsection number (usually 3.2em)

%-----------------------------------------------------
%设置页面样式
%\geometry{top=3.5cm,bottom=3.5cm}
%\linespread{1.15}
\geometry{hcentering}
\pagestyle{fancy}
\setlength{\headheight}{15.0pt}
\fancyhf{}
\fancyhead[LE,RO]{\thepage}
\fancyhead[RE]{\bf\nouppercase{\rightmark}}
\fancyhead[LO]{\bf\nouppercase{\leftmark}}
\renewcommand{\headrulewidth}{0pt}
\renewcommand{\footrulewidth}{0pt}

\def\sub{\subseteq}
\def\sups{\supseteq}
\def\emp{\varnothing}
\def\GL{\mathrm{GL}}
\def\End{\mathrm{End}}
\def\Hom{\mathrm{Hom}}
\def\Aut{\mathrm{Aut}}
\def\det{\mathrm{det}}
\def\Ext{\mathrm{Ext}}
\def\tr{\mathrm{tr}}
\def\O{\mathrm{O}}
\def\ker{\mathrm{ker}\,}
\def\coker{\mathrm{coker}\,}
\def\im{\mathrm{im}\,}
\def\coim{\mathrm{coim}\,}
\def\rank{\mathrm{rank}\,}
\def\char{\mathrm{char}\,}
\def\supp{\mathrm{supp}\,}
\def\sign{\mathrm{sign}}
\def\Int{\mathrm{Int}}
\def\id{\mathrm{id}}
\def\pr{\mathrm{pr}\,}
\def\St{\mathrm{St}\,}
\def\st{\mathrm{st}\,}
\def\C{\mathbb{C}}
\def\K{\mathbb{K}}
\def\F{\mathbb{F}}
\def\R{\mathbb{R}}
\def\B{\mathbb{B}}
\def\Q{\mathbb{Q}}
\def\Z{\mathbb{Z}}
\def\N{\mathbb{N}}
\def\T{\mathbb{T}}
\def\H{\mathbb{H}}
\def\RP{\mathbb{R}\mathrm{P}}
\def\CP{\mathbb{C}\mathrm{P}}
\def\P{\mathbb{P}}
\def\eps{\varepsilon}
\def\And{\quad\textit{and}\quad}
\def\for{\quad\textit{for}\quad}
\makeatletter
\@addtoreset{equation}{section}
\makeatother
\renewcommand{\theequation}{\arabic{section}.\arabic{equation}}

\makeatletter
\newcommand{\rmnum}[1]{\romannumeral #1}
\newcommand{\Rmnum}[1]{\expandafter\@slowromancap\romannumeral #1@}
\makeatletter
\let\save@mathaccent\mathaccent
\newcommand*\if@single[3]{%
	\setbox0\hbox{${\mathaccent"0362{#1}}^H$}%
	\setbox2\hbox{${\mathaccent"0362{\kern0pt#1}}^H$}%
	\ifdim\ht0=\ht2 #3\else #2\fi
}
%The bar will be moved to the right by a half of \macc@kerna, which is computed by amsmath:
\newcommand*\rel@kern[1]{\kern#1\dimexpr\macc@kerna}
%If there's a superscript following the bar, then no negative kern may follow the bar;
%an additional {} makes sure that the superscript is high enough in this case:
\newcommand*\widebar[1]{\@ifnextchar^{{\wide@bar{#1}{0}}}{\wide@bar{#1}{1}}}
%Use a separate algorithm for single symbols:
\newcommand*\wide@bar[2]{\if@single{#1}{\wide@bar@{#1}{#2}{1}}{\wide@bar@{#1}{#2}{2}}}
\newcommand*\wide@bar@[3]{%
	\begingroup
	\def\mathaccent##1##2{%
		%Enable nesting of accents:
		\let\mathaccent\save@mathaccent
		%If there's more than a single symbol, use the first character instead (see below):
		\if#32 \let\macc@nucleus\first@char \fi
		%Determine the italic correction:
		\setbox\z@\hbox{$\macc@style{\macc@nucleus}_{}$}%
		\setbox\tw@\hbox{$\macc@style{\macc@nucleus}{}_{}$}%
		\dimen@\wd\tw@
		\advance\dimen@-\wd\z@
		%Now \dimen@ is the italic correction of the symbol.
		\divide\dimen@ 3
		\@tempdima\wd\tw@
		\advance\@tempdima-\scriptspace
		%Now \@tempdima is the width of the symbol.
		\divide\@tempdima 10
		\advance\dimen@-\@tempdima
		%Now \dimen@ = (italic correction / 3) - (Breite / 10)
		\ifdim\dimen@>\z@ \dimen@0pt\fi
		%The bar will be shortened in the case \dimen@<0 !
		\rel@kern{0.6}\kern-\dimen@
		\if#31
		\overline{\rel@kern{-0.6}\kern\dimen@\macc@nucleus\rel@kern{0.4}\kern\dimen@}%
		\advance\dimen@0.4\dimexpr\macc@kerna
		%Place the combined final kern (-\dimen@) if it is >0 or if a superscript follows:
		\let\final@kern#2%
		\ifdim\dimen@<\z@ \let\final@kern1\fi
		\if\final@kern1 \kern-\dimen@\fi
		\else
		\overline{\rel@kern{-0.6}\kern\dimen@#1}%
		\fi
	}%
	\macc@depth\@ne
	\let\math@bgroup\@empty \let\math@egroup\macc@set@skewchar
	\mathsurround\z@ \frozen@everymath{\mathgroup\macc@group\relax}%
	\macc@set@skewchar\relax
	\let\mathaccentV\macc@nested@a
	%The following initialises \macc@kerna and calls \mathaccent:
	\if#31
	\macc@nested@a\relax111{#1}%
	\else
	%If the argument consists of more than one symbol, and if the first token is
	%a letter, use that letter for the computations:
	\def\gobble@till@marker##1\endmarker{}%
	\futurelet\first@char\gobble@till@marker#1\endmarker
	\ifcat\noexpand\first@char A\else
	\def\first@char{}%
	\fi
	\macc@nested@a\relax111{\first@char}%
	\fi
	\endgroup
}
\makeatother
\renewcommand{\thefigure}{\arabic{chapter}.\arabic{figure}}
\renewcommand{\theequation}{\arabic{section}.\arabic{equation}}
\makeatletter

\makeatletter
\newcommand{\colim@}[2]{%
	\vtop{\m@th\ialign{##\cr
			\hfil$#1\operator@font colim$\hfil\cr
			\noalign{\nointerlineskip\kern1.5\ex@}#2\cr
			\noalign{\nointerlineskip\kern-\ex@}\cr}}%
}
\renewcommand{\varprojlim}{%
	\mathop{\mathpalette\varlim@{\leftarrowfill@\scriptscriptstyle}}\nmlimits@
}
\renewcommand{\varinjlim}{%
	\mathop{\mathpalette\varlim@{\rightarrowfill@\scriptscriptstyle}}\nmlimits@
}
\makeatother

\def\llim{\varprojlim\limits}
\def\rlim{\varinjlim\limits}